\documentclass{article}
\usepackage{algpseudocode}
\usepackage{algorithm}
\usepackage[english]{babel}
\usepackage[utf8]{inputenc}
\usepackage{fancyhdr}
\usepackage{listings}
\usepackage{graphicx}
\usepackage[section]{placeins}
\usepackage{titlesec}
\usepackage{amssymb}
\usepackage{amsmath}
\usepackage[margin=1in]{geometry}
\pagestyle{fancy}
\fancyhf{}
\rhead{Rahul Shah}
\lhead{Machine Learning: Nanodegree}
\rfoot{\thepage}
\title{Supervised Learning: Perceptron Algorithm}
\begin{document}
\maketitle
\section{Motivation}
A college's acceptances are a yes/no decision. The factors the college considers are:\\
$x_1:=$ test score / 10\\
$x_2:=$ grades / 10\\

Can we predict whether a new $(x_1, x_2)$ gets accepted or rejected?

\section{Classification Problem}
This is a classification problem, so our goal is not to fit a function to the data, but instead fit a function such that it splits the data into discrete, already defined categories. In the case of the college problem, we plot the points on an $x_1$ vs $x_2$ plot, and come up with the equation of a line $f(x_1.x_2) = \beta_2 x_2 + \beta_1 x_1 + \beta_0$ that splits the data in half. Then, for any given $(x_1,x_2)$, our prediction equation is: \\
$\hat{y} = 
\begin{cases} 
0 & f(x_1, x_2) \le 0 \\
1 & f(x_1, x_2) > 0 
\end{cases}
$\\
Generally, if each point on our plot has n coordinates, our dividers are going to be n-planes, or an n-1 subspace of $\mathbb{R}^n$, and our $f(x_1, x_2) = \beta^T X + \beta_0$, where both $\beta$ and $X$ are column vectors containing the respective elements. Just like linear regression, we can also bend our plane in polynomial shapes by adding new terms of the form $x_{i}^n$
\section{Perceptron}
%This is best done with a picture

\subsection{Another Way}
% again, picture
\subsection{Logical Operators}
Basically, if we plot points at $(0, 0), (0, 1), (1, 0), (1, 1)$, we need to come up with a dividing line or whatever such that all points above it, when plugged in, return true, and all other points return false. Try plugging in these points into equations of the form $f(x_1.x_2) = \beta_2 x_2 + \beta_1 x_1 + \beta_0$ that follow the rules listed for each logical operator and compare the output with the logical operator it's supposed to represent:\\
AND: $\beta_0 < 0, \beta_1 + \beta_0 < 0, \beta_2 + \beta_0 < 0, \beta_2 + \beta_1 + \beta_0 > 0$ \\
OR: $\beta_0 < 0, \beta_1 + \beta_0 > 0, \beta_2 + \beta_0 > 0, \beta_2 + \beta_1 + \beta_0 > 0$ \\
NOT: $\beta_0 > 0, \beta_1 + \beta_0 < 0$

You can compose the rest of these with one another to get other functions.

\section{Finding the Constant Terms}
Now that you know both the 2 variable and the general case of this algorithm, we can use what we've learned so far to figure out what the constant terms $\beta$ are (ya know, the whole point of this algorithm).
\subsection{Perceptron Trick}
\subsubsection{Intuition}
For any misclassified point, we want to move the line closer to the misclassified points, because the line is the border between our classification 1 or 0, so crossing that line equates to changing the predicted value. Kind of like how we do the absolute trick, where we want the line to move to a place where it approximately fits with the data, we want to do a similar thing here. But here our line (or hyperplane) acts as a border between points, not as a representation of the data. So, we want the "broad side" of the line to move towards the points that are misclassified, so we flip all our $\pm$ signs to $\mp$ signs. 
% THIS IS A BAD EXPLANATION FOR THIS, i dont think it hits the exact reason why we do what we're doing this. Why, here, do we apply the + and -s simultaneously here but not for the absolute trick? This probably requires playing around with numbers.
\subsubsection{Equation}
So for each misclassified point,\\
$f(\beta, \alpha, X, P) = (\beta - \alpha P)^TX$\\
Where:\\
$\beta :=$ Constants in vector form, including $\beta_0$\\
$\alpha :=$ learn rate\\
$X :=$ the data input (appended with a 1 for the constant term $\beta_0$)\\
$P :=$ the point that is misclassified $(p_1, p_2, ...)$\\

\subsection{Algorithm}
In general, the steps are as follows:

\begin{itemize}
	\item Start with random weights
	\item figure out what points are misclassified (apply current linear eqn to the dataset and see if each corresponding data point matched with the actual data point's classification)
	\item For each point that is misclassified, change weights by the value of their corresponding x multiplied by the learn rate
\end{itemize}

\begin{algorithm}
	\caption{Perceptron Algorithm}
	\begin{algorithmic}
		\Procedure{Step}{$X$, $\beta$, $y$, $\alpha$}\\
		\Comment $X$ the data input, $\beta$ the coefficients to the linear eqn, $y$ the actual result values of the predictor output, $\alpha$ is the learning rate
		
		\State $\hat{y} = \beta^TX$
		
		\For { each point in (X, y)}\\\\
		\Comment Remember, when we read in the data, it's read in as ($<$input$>$, $<$output$>$), where $X = $ input (n col matrix) and $y = $ output (vector)\\
		
		\If {$\hat{y} == 0$ and the corresponding y is different}:
			\State $\beta += \alpha X$
		\EndIf\\
		
		\If {$\hat{y} == 1$ and the corresponding y is different}:
			\State $\beta -= \alpha X$
		\EndIf\\
		
		\EndFor
		
		
		\EndProcedure
	\end{algorithmic}
\end{algorithm}



\end{document}