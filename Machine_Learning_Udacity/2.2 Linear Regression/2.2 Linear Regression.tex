\documentclass{article}
\usepackage[english]{babel}
\usepackage[utf8]{inputenc}
\usepackage{fancyhdr}
\usepackage{listings}
\usepackage{graphicx}
\usepackage[section]{placeins}
\usepackage{titlesec}
\usepackage[margin=1in]{geometry}
\pagestyle{fancy}
\fancyhf{}
\rhead{Rahul Shah}
\lhead{Machine Learning Udacity}
\rfoot{\thepage}
\title{Supervised Learning: Linear Regression}
\begin{document}
\maketitle
\section{Motivation}
% Talk about the housing problem and trying to predict the next house in
\section{Fitting a Line}
% If we know the thing is linear, then we can use a linear model to predict it
% make sure you mention the linear model we are using and what the coefficients mean
\section{Moving a Line Towards a Point}
% This is how we can approximate a line movement
\subsection{Absolute Trick}
\subsection{Square Trick}

% Think of each point pulling towards a line, the farther away the point from the line, the harder it pulls the line towards it
% Note that if we define error as the difference between our predicted y from our actual point, our best fit line is essentially the placement of the line such that the sum of all errors is at its smallest

% Note on Linear Lease Squares (matrix solution), and how, though it does solve the problem completely, it also involves inverting a matrix, which can be computationally taxing
% So, we take an iterative approach, (mention the mountain thing, and how, to get to the bottom, you just go where it falls off the most)

\section{Gradient Descent}
% Gradient is basically just the slope of the moutain at any point around you
% Mountain -> error function which we need to minimize
% Define error more precisely, introducing avg error
% MAE & MSE

\subsection{Types of Gradient Descent}
% Batch, Stochastic, and Mini-Batch
% First two are basically terrible
% give pseudocode for mini-batch

\section{Generalizing to Higher Dimensions}
% Absolutely give matrix translations here or else nothing makes any goddamn sense
% Also pseudocode

\section{Warnings}
% Just copy whats on the Udacity site basically

\section{Polynomial Regression}
% you can do something similar if you want to fit to a polynomial

\section{Regularization}

\section{Feature Scaling}

\section{External Resources}

\end{document}